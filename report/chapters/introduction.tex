\section*{Introduction}
Medical images are widely used for all kinds of diagnoses such as tumour detection. Manual tissue segmentation is tedious and repetitive tasks for the medical personnel. A medical image analysis (MIA) pipeline is a tool that can reduce the time needed to perform said tasks. The current trend of machine learning offers new possibilities for this tool, especially in the classification step.
This work aims to test the following hypothesis: \textit{Atlas based segmentation consists of a powerful baseline for brain tissue segmentation when compared to an ML based approach}. To accept or reject this hypothesis, different atlas-based segmentations were implemented and compared with machine learning.
In this project, four variations of atlas-based segmentation and one machine learning algorithm are compared. Those are majority voting, global and local weighted voting and shape based averaging for atlas-based segmentation versus random forest for the machine learning segmentation. The comparison is done with a Dice score and a Hausdorff distance, two widely used measurement tools. The same set of anonymized images have been used for the training and testing of the different methods. The registration quality have an obvious impact on an atlas-based segmentation performance. Intuitively, if the target image and the atlas image are the same, the only source of error is the segmentation. An non-rigid registration allows to reduce the dissimilarity between the target image and the atlas. Thus two kinds of registration have been implemented and compared: affine and non-rigid.

\subsection*{Related work}
Organ segmentation is a big concern in medical image processing, and it exists numerous works on the subject. It has been shown that multi-atlas segmentation performs better than single-atlas for brain tissue segmentation \cite{Klein2005}. Thus, no single-atlas method was used in this work. Multi-atlas fusion, including majority voting, local and global weighting strategies, and the combination of those methods are studied in \cite{Artaechevarria2009}.
Shape-based averaging method for multi-atlas fusion has been introduced by \cite{Rohlfing2007}.
Each of the atlas-based methods implemented in this work for segmentation are described in detail in those papers.